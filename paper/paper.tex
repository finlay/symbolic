\documentclass{amsart}

\title{Symbolic experiment}
\author{Finlay Thompson}

\providecommand{\Z}{\mathbb{Z}}
\providecommand{\R}{\mathbb{R}}
\providecommand{\HH}{\mathbb{H}}
\providecommand{\so}{\mathfrak{so}}
\renewcommand{\P}{\underline{P}}
\newcommand{\Q}{\underline{Q}}
\newcommand{\End}{\mathrm{End}}


\begin{document}
\maketitle

\section{Quaternions}

The algebra of quaternions, $\HH$, naturally arises from the classification of
central simple algebras over the field of real numbers, $\R$. It appears as the
generator of the Brauer group of the real numbers:
$$ Br(\R) = \Z_2 = \{\R, \HH\} $$
Another way of expressing this is that the tensor product of the quaternions
with themselves is a real matrix algebra.
$$ \HH\otimes\HH = M(4, \R) $$

More traditionally the algebra of quaternions is presented as the real associate
algebra spanned by four symbols $\{1,i,j,k\}$, with the following relationships,
$$ i^2 = j ^2 = k^2 = ijk = -1 $$

The imaginary part of the quaternions, or the span of $\{i,j,k\}$, can be
identified with the Lie algebra of three dimensional rotations, $\so(3)$, or
with the standard three dimensional vector calculus of dot and cross products.
If we write,
$$ \HH = \R \oplus \so(3) $$
then the product is expressed as:
\begin{align*}
  p \cdot q &= (p_0 + \P) \cdot (q_0 + \Q) \\
            &= p_0q_0 - \P\cdot \Q + p_0\Q - q_0\P + [\P,\Q] \\
            &= p_0q_0 - \P\cdot \Q + p_0\Q - q_0\P + \P\times \Q
\end{align*}


\section{Quaternion value functions}

Let $U\subset\HH$ be a contractible open subset of $\HH$. In this section we are
interested in smooth functions $f : U \to \HH$. As a real function $f$ has four
components, each depending on four local variables.

The usual definition of the differential $df$ is given as:
\begin{align*}
  df &: \mathfrak{X}(U) \to \HH \\
     &: X \mapsto X(f)
\end{align*}
Where $X \in \mathfrak{X}(U)$ is a vector field defined on $U$. However, because
$U$ is a contractible open subset, we can identify the tangent space with $\HH$:
$$ TU = U \times \HH $$
And the vector fields as:
\begin{align*}
\mathfrak{X}(U)
    &= \Gamma(TU) \\
    &= \Gamma(U \times \HH) \\
    &= U \to \HH
\end{align*}

So, using the fact that the category of real vector spaces is Cartesian closed,
the differential $df$ becomes:
\begin{align*}
  df &: \mathfrak{X}(U) \to \HH \\
     &: (U \to \HH) \to \HH \\
     &: U \to \End(\HH) \\
     &: U \to \HH \otimes \HH^\circ
\end{align*}

%\emph{Some synthetic differential geometry here!}


\subsection{The Brauer isomorphism}

The last step makes use of the Brauer isomorphism,
$$ \End(\HH) = M(4, \R) = \HH \otimes \HH^\circ $$
which is defined in terms of elements $x,p,q\in\HH$ as:
$$ (p\otimes q)(x) = p \cdot x \cdot q $$
Notice that we need to use the opposite algebra on the right, which is
isomorphic to to $\HH$, but has the product written the other way around.

The Brauer isomorphism tells us that \emph{all} endomorphisms can be written
this way, and so it is natural to ask what the components of the function $df$
are, when expressed as an $\HH\otimes\HH^\circ$ valued function on $U$.

The isomorphism can be written in local variables. We will use $x\in U\subset
\HH$ as local variable in the domain space, and indexes for the individual
components.
\begin{align*}
  x &= x_0 + \vec{X} \\
    &= x_0\boldsymbol{e}^0 + x_1\boldsymbol{e}^1 + x_2\boldsymbol{e}^2 + x_3\boldsymbol{e}^3 \\
    &= x_0\boldsymbol{1} + x_1\boldsymbol{i} + x_2\boldsymbol{j} + x_3\boldsymbol{k} \\
\end{align*}
where the $x_i$ are real numbers, and $\vec{X}$ is an element of $\so(3)$.

Any linear map $A \in \End(\HH) $ can be written
\begin{align*}
  A &= \sum_{\mu\nu} a_\mu^\nu \boldsymbol{e}^\mu \otimes \boldsymbol{\epsilon}_\nu \\
    &= \begin{pmatrix}
      a_0^0 & a_1^0 & a_2^0 & a_3^0 \\
      a_0^1 & a_1^1 & a_2^1 & a_3^1 \\
      a_0^2 & a_1^2 & a_2^2 & a_3^2 \\
      a_0^3 & a_1^3 & a_2^3 & a_3^3 \\
    \end{pmatrix}
\end{align*}

\end{document}
